% Chapter Template

\chapter{Conclusiones} % Main chapter title

\label{Chapter5} % Change X to a consecutive number; for referencing this chapter elsewhere, use \ref{ChapterX}
En este capitulo se presentan las conclusiones relativas al alcance de los objetivos planteados.
También se presentan los próximos pasos y cambios para lograr un instrumento totalmente funcional a la producción científica del instituto.

%----------------------------------------------------------------------------------------

%----------------------------------------------------------------------------------------
%	SECTION 1
%----------------------------------------------------------------------------------------

\section{Conclusiones generales }

El trabajo cumplió de forma satisfactoria la puesta en funcionamiento del espectrofotometro, después de haber estado mucho tiempo sin funcionar.
Esta puesta en servicio genero una expectativa de parte del laboratorio, ya que este equipo tiene un sistema optico de muy alta calidad y en perfecto estado.
Esto es un valor agregado al proyecto ya que existen en la actualidad muchos instrumentos que se encuentran en desuso por falta de repuestos o de desarrollos como este.


\begin{itemize}
\item Se pudo adaptar mecánicamente un motor paso a paso al selector de longitud de onda.Se realizaron pruebas de funcionalidad, seleccionando y posicionando el mismo en diferentes longitudes de onda.
   
\item  
\item Capítulo 3: Diseño e implementación
\item Capítulo 4: Ensayos y resultados
\item Capítulo 5: Conclusiones
\end{itemize}



%La idea de esta sección es resaltar cuáles son los principales aportes del trabajo realizado y cómo se podría continuar. Debe ser especialmente breve y concisa. Es buena idea usar un listado para enumerar los logros obtenidos.%

%----------------------------------------------------------------------------------------
%	SECTION 2
%----------------------------------------------------------------------------------------
\section{Próximos pasos}

%Acá se indica cómo se podría continuar el trabajo más adelante.
